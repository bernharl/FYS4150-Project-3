\documentclass[twocolumn]{aastex62}
\newcommand{\vdag}{(v)^\dagger}
\newcommand\aastex{AAS\TeX}
\newcommand\latex{La\TeX}
\usepackage{amsmath}
\usepackage{physics}
\usepackage{hyperref}
\usepackage{natbib}
\usepackage[T1]{fontenc}
\usepackage[english]{babel}
\usepackage[utf8]{inputenc}

\begin{document}

\title{Solving Eigenvalue Problems by Means of the Jacobi Algorithm}

\author{Håkon Tansem}

\author{Nils-Ole Stutzer}

\author{Bernhard Nornes Lotsberg}

\begin{abstract}
\end{abstract}

\section{Introduction} \label{sec:intro}
When solving problems in science and mathematics an ever recurring problem is to solve integrals. Integrals are found in all sorts of manners, e.g. when computing expectation values in quantum mechanics. In this paper we will consider ways of solving an example of a six dimensional expectation value problem from quantum mechanics integral in several different ways. We will consider a brute force Gauss-Legendre quadrature, an improved Gauss-Laguerre quadrature, a brute force monte carlo integration and a monte carlo integration with importance sampling as shown by \cite{press:2007} and \cite{jensen:2015}. The resulting integrals will be compared to the analytical solution and the run times are compared, in order to find which method is most efficient.

In the theory section we present needed theory wich we discuss how to implement in the method section. The results are presented in the results section and discussed in the discussion section.

\section{Theory} \label{sec:theory}
\subsection{The integral}
Before stating the needed integration methods we used we present the integral to integrate. Assuming the wave function of two electrons can be modelled as a single-particle wave function in a hydrogen atom, the wave function of the $i$th electon in the 1 $s$ state is given by 
\begin{align}
	\psi_{1,s}(\vec{r}_i) = e^{-\alpha r_i},
\end{align}
where the dimensionless position
\begin{align}
	\vec{r}_i = x_i \hat{e}_x + y_i\hat{e}_y + z_i\hat{e}_z
\end{align} 
with orthogonal unit vectors $\hat{e}_i$.

The distance $r_i = \sqrt{x_i^2 + y_i^2 + z_i^2}$ and we let the parameter $\alpha = 2$ corresponding to the charge of a helium $Z = 2$. Then the ansats for the wave function for two electrons is given by the product of the two 1 $s$ wave functions 
\begin{align}
	\Psi(\vec{r}_1, \vec{r}_2) = e^{-\alpha(r_1 + r_2)}.
\end{align}
We now want to find the expectation value of the correlation energy between electrons which repel each other by means of the Coulomb interaction as 
\begin{align}
\langle \frac{
1}{\vec{r}_1 - \vec{r}_2}\rangle = \int d\vec{r}_1d\vec{r}_2 e^{-2\alpha(r_1 + r_2)}\frac{1}{|\vec{r}_1 - \vec{r}_2|}.
\label{eq:integral}
\end{align}
This integral has an analytical solution $5\pi^2/16^2$, which we can later compare numerical results to.
\subsection{Brute Force Gauss-Legendre Quadrature}
The following theory and derivations of the integration methods presented follows closely \citep[Ch. 5.3]{jensen:2015}. 

The essence of Gaussian Quadrature (GQ) is to approximate an integral 
\begin{align}
	I = \int f(x) dx \approx \sum^N_{i = 1} \omega_i f(x_i),
	\label{eq:quadrature}
\end{align} 
for some weights $\omega_i$ and grid points $x_i$. The grid points and wights are obtained through the zeros of othogonal polynomials. These polynomials are orthogonal on some intervale, for instance $[-1, 1]$ for Legendre polynomials. Since we must find $N$ grid points and weigths, we must fit $2N$ parameters. Therefore we must approximate the integrand $f(x)$ by a polynomial of degree $2N-1$, i.e. $f(x) \approx P_{2N-1}(x)$. Then the integral 
\begin{align}
	I \approx \int P_{2N-1}(x)dx = \sum^{N-1}_{i=0}P_{2N-1}(x_i) \omega_i.
\end{align} 
GQ can integrate all polynomials of degree up to $2N-1$ exactelly, we thus get an equallity when approximating the integral of $P_{2N-1}$. 

If we choose to expande the polynomial $P_{2N-1}$ in terms of the Legendre polynomials $L_N$, we can through polynomial devision write 
\begin{align}
	P_{2N-1} = L_N(x)P_{N-1}(x) + Q_{N-1}(x),
\end{align}
where $P_{N-1}$ and $Q_{N-1}$ are polynomials of degree $N-1$. We can thus write 
\begin{align}
	I \approx \int^1_{-1} P_{2N-1}(x) dx &= \int^1_{-1} (L_N(x)P_{N-1}(x) + Q_{N-1}(x))dx \\
	&= \int^1_{-1}Q_{N-1}(x)dx,
\end{align}
where the last equallity is due to the orthogonallity between $L_N(x)$ and $P_{N-1}(x)$. Furthermore, $P_{2N-1}(x_k) = Q_{N-1}(x_k)$ for the zeros $x_k$ ($k = 0, 1, 2,\ldots, N-1$) of $L_N$, we can fully define the polynomial $Q_{N-1}(x)$ and thus the integral. The polynomial $Q_{N-1}(x)$ can further be expanded in terms of the Legendre basis 
\begin{align}
	Q_{N-1}(x) = \sum^{N-1}_{i=0} \alpha L_i(x).
	\label{eq:Qexpansion_of_x}
\end{align}
When integrating this we get
\begin{align}
	\int^1_{-1}Q_{N-1}(x)dx = \sum^{N-1}_{i=0} \alpha_i\int^1_{-1}L_0(x)L_i(x) dx = 2\alpha_0,
\end{align}
where we insert that the first Legendre polynomial is normalized to $L_0 = 1$ and utilize the orthogonality relation of the Legendre basis.

Since we know the value of $Q_{N-1}(x)$ at the zeros of $L_N(x)$, we can rewrite (\ref{eq:Qexpansion_of_x})
\begin{align}
	Q_{N-1} (x_k)= \sum^{N-1}_{i=0} \alpha L_i(x_k).
	\label{eq:Qexpansion}
\end{align}
The resulting matrix $L_i(x_k) = L_{ik}$ has linearly independent columns due to the Legendre polynomials being linearly independent as well. Therefore the matrix $L_{ik}$ is orthogonal, i.e.
\begin{align}
	L^{-1}L = I.
\end{align}
We thus multiply both sides of (\ref{eq:Qexpansion}) by $\sum^{N-1}_{i=0}L^{-1}_{ij}$, so that 
\begin{align}
	\alpha_k = \sum_{i=0}^{N-1} (L^{-1})_{ki}Q_{N-1}(x_i).
\end{align}
This result in addition to the approximation of the integral then gives
\begin{align}
	I &\approx \int^1_{-1} P_{2N-1}(x)dx = \int^1_{-1} Q_{N-1}(x)dx = 2\alpha_0 \\
	&= 2 \sum^{N-1}_{i=0} (L^{-1})_{0i}P_{2N-1}(x_i).
	\label{eq:int_approx}
\end{align}
Here we can clearly see that the weights $\omega_i = 2(L^{-1})_{0i}$ and the meshpoints $x_i$ are the zeros of $L_N(x)$. Finding the weights is now simply a matrix inversion problem and the meshpoints can be found by for instance Newtons method. Thus (\ref{eq:int_approx}) is on the same form as (\ref{eq:quadrature}), yielding an approximation to $I$. When performing an integral with more general limits $[a,b]$, one can now perform a simple change of variables $\tilde{x} = \frac{b - a}{2}x + \frac{b + a}{2}$, to accomodate for this. Note that when considering the Gauss-Legendre approach, and we integrate (\ref{eq:integral}) using cartesian coordinates, we must let the limits of all six cartesian itegrals $a = -\lambda$ and $b = \lambda$ where $\lambda\to\infty$.

When integrating a six dimensional integral like the quantum mechanical problem stated earlier, we can simply use Gauss-Legendre quadrature on all six integrals in a brute force way. Then the integral simply becomes 
\begin{align}
	I &= \int d\vec{r}_1d\vec{r}_2 e^{-2\alpha(r_1 + r_2)}\frac{1}{|\vec{r}_1 - \vec{r}_2|} \\
	&\approx \sum_{i, j, k, l, m, n = 0}^{N-1} \omega_i \omega_j \omega_k \omega_l \omega_m \omega_n f,
\end{align} 
where $f = f(x_1^i, y_1^j, z_1^k, x_2^l, y_2^m, z_2^n)$ denotes the integrand and each cartesian variable $x_i$ has the same weights $\omega$ due to identical integration limits on the six integrals. 

\subsection{Improved Gaussian Quadrature}
The above meansioned Gauss Laguerre quadrature is in fact quite inacurate in the case of our integral. In order to improve on the method, one can use a different orthogonal polynomial basis. In our case since we have an integral on the form
\begin{align}
	I = \int^\infty_0 f(x)dx = \int^\infty_0x^2e^{-x}g(x) dx
	\label{eq:laguerre_integral}
\end{align}
it is in fact way more efficient to use Laguerre polynomials as oppose to Legendre polynomials. Then when finding an approximation to the integral the $x^2e^{-x}$ factor of the integrand is absorbed into the weights $\omega_i$. Thus the approximation becomes more accurate, as we get more suitable weights. The derivation of the weights is however not shown here as it is completely analogous to the derivation shown for the Gauss-Legendre quadrature.

In order to transorm our integrand to a form resembeling (\ref{eq:laguerre_integral}) we need to transform from cartesian to spherical coordinates, i.e. $(x, y, z)\to(r, \theta, \phi)$ where $r\in[,\infty)$, $\theta\in[0,\pi]$ and $\phi\in[0,2\pi]$. In spherical coordinates the differential we get that the differential and the relative distance between the electrons become 
\begin{align}
	d\vec{r}_1d\vec{r}_2 &= r_1^2r_2^2 dr_1dr_2\sin(\theta_1)\sin(\theta_2)d\theta_1d\theta_2d\phi_1d\phi_2\\
	|\vec{r}_1 - \vec{r}_2| &= \sqrt{r_1^2 + r_2^2 - 2r_1r_2cos(\beta)},
\end{align}
where $cos(\beta) = cos(\theta_1)\cos(\theta_2) + \sin(\theta_1)\sin(\theta_2)\cos(\phi_1 - \phi_2)$.
Next we introduce the change of variables $u = \alpha r \implies du = \alpha dr$ so that the integral is on the form
\begin{align}
	I = \frac{1}{32 \alpha^5} \int^\pi_0\int^{\pi}_0\int^{2\pi}_0\int^{2\pi}_0\int^\infty_0\int^\infty_0 fdu_1du_2d\theta_1d\theta_2d\phi_1d\phi_2,
\end{align}
where $f = f(u_1, u_2, \theta_1, \theta_2, \phi_1, \phi_2) = \frac{\sin
	(\theta_1)\sin(\theta_2)u_1^2u_2^2e^{-(u_1+u_2)}}{\sqrt{u_1^2 + u_2^2 - 2u_1u_2cos(\beta)}}$ is the new integrand.
Now, for the radial part ($u_1$ and $u_2$) a Gauss-Laguerre approach is used, while we use Gauss-Legendre quadrature for the angular part. 

\subsection{Brute Force Monte Carlo Integration}
An integration method frequently used, espessially when computing multi-dimensional integrals as its error remains constant for any higher dimension, is the Monte Carlo integration. In this integration method one approximates the integral by an expectation value. Consider for instance an integral 
\begin{align}
	I &= \int^b_a f(x)dx = (b-a)\int^b_a\frac{f(x)}{b-a}dx \\	
	&= (b-a)\int^b_af(x)p(x)dx,
\end{align}
where we let $p(x) = \frac{1}{b-a}$ be the uniform probability density function PDF for stochastic variables $x\in[a, b]$. Since we know that an expectation value 
\begin{align}
	\langle f(x)\rangle = \int^b_a f(x)p(x)dx \approx \frac{1}{N}\sum_{i=0}^{N-1} f(x_i),
\end{align}
where we used that the expectation value of $f(x)$ is approximately the average of the $f(x_i)$'s where the $x_i$'s are drawn from the distribution $p(x)$, for large enough sample size $N$. 

Note that we implisitly performed a mapping in this case. Because uniform distribution by default only returns values $y\in[0, 1]$, we change the variable to $x = a + (b-a)y$ so as to draw values $x_i$ from a uniform distribution between $a$ and $b$. 

The integral can thus be approximated by 
\begin{align}
	I \approx (b-a)\langle f(x) \rangle \approx \frac{b-a}{N}\sum^{N-1}_{i=0}f(x_i).
\end{align}
In order to get an estimate for the accuracy of the integration we can calculate the variance defined as 
\begin{align}
	\sigma^2 = \frac{1}{N}\sum_{i=0}^{N-1} f^2(x_i) - \left(\frac{1}{N}\sum_{i=0}^{N-1}f(x_i)\right)^2 = \langle f^2\rangle - \langle f\rangle^2.
\end{align}
As a Monte Carlo integration is a statistical experiment, the variance $\sigma^2$ is a measure of the spread from the mean of the integral approximation. We thus whant to minimize this quatity.

When solving (\ref{eq:integral}) using cartesian coordinates, we use the same integration limits as in the Gauss-Legendre quadrature. The we simply get a six-dimensional expectation value so that 
\begin{align}
	I = (b-a)^6\langle f \rangle \approx \frac{(b-a)^6}{N}\sum^{N-1}_{i=0} f(x_1^i, y_1^i, z_1^i, x_2^i, y_2^i, z_2^i),
\end{align}
where the cartesian coordiantes $x^i$ are all drawn from a uniform distribution for $x^i\in[-\lambda, \lambda]$. The variance is calculated completely analogous to the one-dimensional case. 

\subsection{Improved Monte Carlo Integration}
One way to improve the Monte Carlo Integration shown in the previous subsection, i.e. to reduce its variance, is to use a different PDE, that fits the shape of the integrand better, to draw the samples from.

If we consider the quantum mechanical integral in spherical coordinates, as shown previously, we recognize the $e^{-u}$ factors as an exponential distribution. Thus we let $p(y) = e^{-y}$ denote the exponential distribution. Using conservation of probability under change of variable, we set $p(y)dy = exp(-y)dy = p(x)dx = dx$ for the uniform PDF $p(x)$ with $x\in[0,1]$. If we integrate this cumulative distribution we find 
\begin{align}
	x(y) = \int^y_0 \exp(-\xi)d\xi = 1 - \exp(-y),
\end{align} 
which we can invert to get the mapping $y(x) = -\ln(1-x)$ from the uniform to the exponential PDF. Now $u = y\in[0\infty)$ is used for the radial distance. We can now absorbe the exponential part of the integral into the expectation value so that 
\begin{align}
	I &= \int^\infty_0 f(u)du = \int_0^\infty \frac{f(u)}{p(u)}p(u)du \\
	&= \int^1_0 g(u(x))dx \approx \frac{1}{N}\sum_{i=0}^{N-1} g(u(x_i), 
\end{align}
where we used that $p(u) = \exp(-u)$ and that $p(u)du = dx$ to change variables. The samples $u(x_i)$ are now drawn from the exponential distribution $p(u)$. This is what is called importance sampling, which results in a lower variance than using the brute force Monte Carlo integration, because we simply sample from a distribution which fits the integrand better. 

However, note that the integral we want to estimate in spherical coordinates only has limits $[0, \infty)$ for the two radial coordiantes $u_1$ and $u_2$. We must thus sample from the uniform distribution for the angular integrals, as the angles exist within the finit limits $[0,\pi]$ and $[0, 2\pi]$, while the radial integral should utilize importance sampling from the exponential distribution.

The final integral then looks as follows
\begin{align}
	I \approx \frac{1}{N}\frac{\pi^4}{8\alpha^5}\sum^{N-1}_{i=0}\frac{\sin
	(\theta_1^i)\sin(\theta_2^i)(u_1^i)^2(u_2^i)^2}{\sqrt{(u_1^i)^2 + (u_2^i)^2 - 2u_1^iu_2^icos(\beta_i)}},
\end{align}
where the index $i$ simply represents that the corresponding samples are drawn from their respective PDF.
\section{Method} \label{sec:method}
When approximating the integrals, one may encounter singularities in the integrand when $|\vec{r}_1 - \vec{r}_2|$ becomes too small, which due to the discretization. In order to handle these singularies we simply let the integrand $f=0$ when this happens.

Next when implementing the two GQ integrators, the integration sum approximating the six-dimensional integral are simply calculated in a sixfold loop. When calculating the integration weights and grid points we use the algorithms presented in (\cite{press:2007}) (implemented in the \texttt{weights.cpp}-file). In the brute force cartesian GQ approach as well as the angular integrals in the improved GQ we compute the weights and meshpoints using the provided \texttt{gauleg}-function, since the integration limits are finite. The weights and meshpoints of the radial integral therewhile are produced using the provided \texttt{gauss\_laguerre}-function.
The integrals where calulated with the two GQ methods for different grid sizes $N$ and then compared. In order to produce a satisfactory result using the brute force Gauss-Legendre quadrature we tryed different grid sizes $N$ and infinity approximations $\lambda$.

When implementing the Monte Carlo integrations, we produce the needed psuedo-random numbers using a Mersenne-Twister algorithm, as it has a suffficiently large period. Furthermore the mapping from the uniform PDF with $x\in[0,1]$ to the uniform PDF with $x\in[a,b]$ and the exponential distributions are done implisitly inside the C++ \texttt{random}-packages used. The Monte Carlo integrals where computed for different sample sizes $N$, different degrees of parallelization and different compiler flags, for comparison. 

\section{Results} \label{sec:results}
\section{Discussion} \label{sec:discussion}
\section{Conclusion} \label{sec:conclusion}

\nocite{jensen:2019}
\bibliographystyle{aasjournal}
\bibliography{ref}

\end{document}

